\documentclass[10pt,conference,compsocconf]{IEEEtran}

\usepackage{hyperref}
\usepackage{graphicx}
\usepackage{xcolor}
\usepackage{blindtext, amsmath, comment, subfig, epsfig }
\usepackage{grffile}
\usepackage{caption}
\usepackage{subcaption}
\usepackage{algorithmic}
\usepackage[utf8]{inputenc}


\title{CS523 Project 1 Report}
\author{Pierre GABIOUD, Justinas SUKAITIS}
\date{April 2020}

\begin{document}

\maketitle

\begin{abstract}
    Please report your design, implementation details, findings of the first project in this report. \\
    You can add references if necessary \cite{article}. \\
    THE REPORT SHOULD NOT EXCEED 3 PAGES.
\end{abstract}
\section{Introduction}
Project 1 aims at implementing \textbf{Secure Multiparty Computation} which is able to do a series of multiple different operations in a private and reliable manner. 
In this project, the different parties get the input computation under the form of \textit{circuits} composed of \textit{gates}, that must be done on every party's values without revealing those values to others. \\

The implementation followed the project description road-map. i.e. We started with implementing the secret sharing phase, followed with the simple operations that do not require additional values \textit{Beaver triplets}: addition, subtraction, multiplication by constant and addition by constant. All of the message passing and secret generation is done in the function \textbf{Run}, while the actual circuit is evaluated in \textbf{gates.go} file in the function evaluate. \\
The multiplication gate however required the use of Beaver triplets: \\
Part 1 assumes the existence of a reliable third party that can not be compromised in order to generate values required for the multiplication operation. \\
Part 2 drops this assumption and uses \textbf{BFV}, which is an implementation of the Fan-Vercauteren version of \textbf{Brakerski's scale invariant homomorphic encryption scheme} for the generation of these values. \\
Both Beaver generator creates an appropriate amount of \textit{Beaver triplets} for each corresponding party, before processing the given circuit.

We note both the \textit{Beaver Triplets} generation and the SMC protocols are entirely separate, where the first protocol only passes the generated values to the second one, in order to respect the Least Common Mechanism paradigm.
\section{Part I}

\subsection{Threat model}
Give the corresponding threat model for the first part of the project that you implemented. 
\subsection{Implementation details}
\begin{itemize}
    \item Give your implementation details
    \item Detail the circuit you created at the end of the first part
\end{itemize}
\section{Part II}
\subsection{Threat model}
Give the corresponding threat model for the second part of the project that you implemented. 
\subsection{Implementation details}
Give implementation details.
\section{Evaluation}
- Give a comprehensive comparison and evaluation about Part1 and Part2 of the project including performance results. Feel free to use charts, tables, plots...\\
\begin{itemize}
    \item What affects the efficiency of the executions? Be specific, which types of operations/circuits are directly linked to performance?
    \item Is there any difference in terms of performance between Part I and Part II? Why? 
\end{itemize}

\section{Discussion}
\begin{itemize}
    \item Comment on your findings, discuss different outcomes for each part.
    \item Discuss outcomes from different circuits including your own circuit.
    \item In your opinion, which model is appropriate to use under which conditions/threat model? Why? Discuss.
    \item Come up with a scenario for each part of the implementation, discuss why it makes sense to use homomorphic encryption based generation of Beaver triplets.
\end{itemize}

\section{Conclusion}
\begin{itemize}
    \item Assess your learning outcomes for this project.
    \item What did you do? What did you learn? Any interesting design ideas? 
\end{itemize}

\bibliographystyle{IEEEtran}
\bibliography{bib}
\end{document}
